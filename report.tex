% Report template for ITV projects
% Based on ABNTEX2 and the thesis template created by Gabriel Garcia (gabrcg@gmail.com)
% This report was created by Hector Azpurua (hector.azpurua@itv.org)

\documentclass[
	% -- opções da classe memoir --
	12pt,				% tamanho da fonte
	openright,			% capítulos começam em pág ímpar (insere página vazia caso preciso)
	oneside,			% para impressão em verso e anverso. Oposto a oneside
	a4paper,			% tamanho do papel. 
	% -- opções da classe abntex2 --
	chapter=TITLE,		% títulos de capítulos convertidos em letras maiúsculas
	%section=TITLE,		% títulos de seções convertidos em letras maiúsculas
	%subsection=TITLE,	% títulos de subseções convertidos em letras maiúsculas
	%subsubsection=TITLE,% títulos de subsubseções convertidos em letras maiúsculas
	% -- opções do pacote babel --
	english,			% idioma adicional para hifenização
	french,				% idioma adicional para hifenização
	spanish,			% idioma adicional para hifenização
	brazil				% o último idioma é o principal do documento
]{abnt/abntex2itv_report}

\usepackage[utf8]{inputenc}		% Codificacao do documento (conversão automática dos acentos)
\usepackage{lastpage}			% Usado pela Ficha catalográfica
\usepackage{indentfirst}		% Indenta o primeiro parágrafo de cada seção.
\usepackage{color}				% Controle das cores
\usepackage{graphicx}			
\usepackage{microtype} 			% para melhorias de justificação
\usepackage{supertabular}       % tabela na capa do documento
\usepackage{booktabs}
\usepackage[table,xcdraw]{xcolor}
\usepackage{adjustbox}
\usepackage{amssymb,amsmath,mathrsfs}
\usepackage{algorithm,algpseudocode}
\usepackage{pgfplots}
\usepackage{tikz}
\usepackage{lipsum}	
\hypersetup{draft}
\usepackage{titlesec}
\usepackage{ragged2e}
\usepackage{tocloft}
\usepackage{etoolbox}

\apptocmd{\thebibliography}{\justifying}{}{} 

\renewcommand{\ABNTEXsectionfont}{\bfseries}

\titlespacing*{\chapter}{0pt}{0pt}{0pt}
\titlespacing*{\section}{0pt}{6pt}{6pt}
\titlespacing*{\subsection}{0pt}{6pt}{6pt}
\titlespacing*{\subsubsection}{0pt}{6pt}{6pt}

% Declaracoes em Português
\algrenewcommand\algorithmicend{\textbf{fim}}
\algrenewcommand\algorithmicdo{\textbf{faça}}
\algrenewcommand\algorithmicwhile{\textbf{enquanto}}
\algrenewcommand\algorithmicfor{\textbf{para}}
\algrenewcommand\algorithmicif{\textbf{se}}
\algrenewcommand\algorithmicthen{\textbf{então}}
\algrenewcommand\algorithmicelse{\textbf{senão}}
\algrenewcommand\algorithmicreturn{\textbf{devolve}}
\algrenewcommand\algorithmicfunction{\textbf{função}}

% Rearranja os finais de cada estrutura
\algrenewtext{EndWhile}{\algorithmicend\ \algorithmicwhile}
\algrenewtext{EndFor}{\algorithmicend\ \algorithmicfor}
\algrenewtext{EndIf}{\algorithmicend\ \algorithmicif}
\algrenewtext{EndFunction}{\algorithmicend\ \algorithmicfunction}

% Espaçamentos entre linhas e parágrafos 
%O tamanho do parágrafo é dado por:
\setlength{\parindent}{1.3cm}
%Controle do espaçamento entre um parágrafo e outro:
\setlength{\parskip}{0.2cm}  % tente também \onelineskip

% ------------------------------------------------%
% Informações de dados para CAPA e FOLHA DE ROSTO %
% ------------------------------------------------%
% \prodtecnica Número de produção técnica ITV
% \titulo Titulo do relatorio Ex: "Relatorio do estado da arte"
% \tiporelatorio Tipo de relatorio: "Final", "Parcial", "de Campo", etc.
% \nomeprojeto Nome do projeto
% \outrossubtitulos Outros subtitulos, pode ser deixado em branco
% \autores Os autores do documento, separados por \\ Ex: Autor 1\\Autor 2\\Autor 3
% \local Lugar onde foi realizado o documento
% \data Data do documento em formato Mês/Ano
% \classificacao Tipo de classificação do documento. Ex: (  ) Confidencial  (X) Restrita  (  )  Uso Interno  (  ) Pública
% \revisao Versão do documento
% \tabelacutter Informação biblioteca ITV
% \palavraschave Palavras chave do documento. Ex: 1. Palavra chave. 2. Palavra chave. 3. Palavra chave.
% \classificacaoassunto Número de Classificação do assunto da dissertação (Consultar biblioteca)
% \parceirologo Logo do parceiro a ser colocado na portada do documento

\prodtecnica{N001 / 2017}
\titulo{Inserir título do relatorio}
\tiporelatorio{(Tipo de relatorio)} 
\nomeprojeto{(Nome do projeto)}
\outrossubtitulos{Outros - identificar de acordo com a lista do CNPq e CAPES.} % opcional
\autores{Autor 1\\Autor 2\\Autor 3}
\local{Ouro Preto\\Minas Gerais, Brasil}
\data{Mês/Ano}
\classificacao{(  ) Confidencial  (X) Restrita  (  )  Uso Interno  (  ) Pública} % Marcar com X a classificação desejada
\revisao{00}
\tabelacutter{000} % Tabela de Cutter (Consultar biblioteca ITV)
\palavraschave{1. Palavra chave. 2. Palavra chave. 3. Palavra chave.}
\classificacaoassunto{000} % Número de Classificação do assunto da dissertação (Consultar biblioteca)
\parceirologo{logos/verlab.png}

% ----------------------------------------%
% Configurações de aparência do PDF final %
% ----------------------------------------%
\definecolor{blue}{RGB}{41,5,195}
\makeatletter
\hypersetup{
     	%pagebackref=true,
		pdftitle={\@title}, 
		pdfauthor={\@author},
    	pdfsubject={\@title},
	    pdfcreator={LaTeX with abnTeX2},
		pdfkeywords={abnt}{latex}{abntex}{abntex2}{\imprimirpalavraschave}, 
		colorlinks=true,       		% false: boxed links; true: colored links
    	linkcolor=blue,          	% color of internal links
    	citecolor=blue,        		% color of links to bibliography
    	filecolor=magenta,      	% color of file links
		urlcolor=blue,
		bookmarksdepth=4
}
\makeatother

\makeindex

\begin{document}

\frenchspacing % Retira espaço extra obsoleto entre as frases.

\imprimircapa

\imprimircatalografica % pode ser comentado

% ---------------- %
% Resumo executivo %
% ---------------- %

\ABNTEXchapterfont\large\textbf{RESUMO EXECUTIVO}

\begin{flushleft}
	
\normalsize
\justify
\normalfont

\lipsum[2] % Comentar e aidicionar resumo executivo aqui

\end{flushleft}
\clearpage

% ----------------- %
% Resumo e abstract %
% ----------------- %

\ABNTEXchapterfont\large\textbf{RESUMO}
\begin{flushleft}
    \normalsize
    \justify
    \normalfont

	
	\lipsum[2] % Comentar e aidicionar resumo aqui
	
\end{flushleft}
\vspace*{1cm}
\ABNTEXchapterfont\large\textbf{ABSTRACT}
\begin{flushleft}
	\normalsize
	\justify
	\normalfont
	
	
	\lipsum[2] % Comentar e aidicionar abstract aqui
	
\end{flushleft}
\clearpage


% ---------------- %
% Lista de figuras %
% ---------------- %

\begin{flushleft}
\ABNTEXchapterfont\Large\textbf{LISTA DE FIGURAS}
\end{flushleft}
\vspace*{-36pt}
\pdfbookmark[0]{\listfigurename}{lof}
\listoffigures*
\cleardoublepage


% ---------------- %
% Lista de tabelas %
% ---------------- %

\begin{flushleft}
\ABNTEXchapterfont\Large\textbf{LISTA DE TABELAS}
\end{flushleft}
\vspace*{-36pt}
\pdfbookmark[0]{\listtablename}{lot}
\listoftables*
\cleardoublepage


% ------------------------------ %
% Lista de siglas e abreviaturas %
% ------------------------------ %
%\vspace*{0.5cm}
%\begin{flushleft}
%\ABNTEXchapterfont\Large\textbf{LISTA DE SIGLAS E ABREVIATURAS}
%    \normalsize
%    \justify
%    \normalfont
%
%\noindent
%AG - Algoritmos Genéticos. \\
%BT - Busca Tabu. 
%\end{flushleft}
%\newpage

% ------- %
% Sumario %
% ------- %

\begin{flushleft}
\ABNTEXchapterfont\Large\textbf{SUMÁRIO}
\end{flushleft}
\vspace*{-36pt}
\pdfbookmark[0]{\contentsname}{toc}
\tableofcontents*
\justify


% -------------------------- %
% Conteúdo do relatorio AQUI %
% -------------------------- %

% Formatação, remover espaço depois dos titulos
\setlength\beforechapskip{-24pt}
\setlength\afterchapskip{12pt}
\textual
\pagestyle{plain}
\normalsize
\justify
\normalfont

\include{sections/full_sections}

% ----------- %
% Referências %
% ----------- %

\titleformat{\chapter}[display]
    {\Large\bfseries}{\vspace*{-36pt}\chaptertitlename\ \thechapter}{12pt}{\Large}
\bibliographystyle{plain}
\bibliography{references}

% -------- %
% Apendice %
% -------- %

\apendices
\justify
\chapter{Título do apêndice}
O apêndice com informações extras

% ------ %
% Anexos %
% ------ %

\anexos
\justify
\chapter{Título do anexo}

Deve ser precedido da palavra ANEXO, identificado por letras maiúsculas consecutivas, travessão e pelo respectivo título. Utilizam-se letras maiúsculas dobradas, na identificação dos anexos, quando esgotadas as letras do alfabeto. 
São considerados ANEXOS: 

Mapas e documentos cartográficos. 

Leis, estatutos e regulamentos que esclareçam as condições jurídicas da pesquisa. 

Textos e reportagens na íntegra.

\end{document}
